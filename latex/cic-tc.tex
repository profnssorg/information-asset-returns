\documentclass[cic,tc, english]{iiufrgs}
%CITAR ESTUDO DE EVENTOS NA INTRODUCAO
%Falar sobre as notícias obtidas em results and discussion
%Inserir todas as notícias no anexo
%Citar o Caldeira

% Pacote Codigo erro
\usepackage{listings}
\lstset{
    frame=single,
    breaklines=true,
    postbreak=\raisebox{0ex}[0ex][0ex]{\ensuremath{\color{red}\hookrightarrow\space}}
}

% Use unicode
\usepackage[utf8]{inputenc}   % pacote para acentuação

\usepackage{amsmath} % para equações
\usepackage{mathtools} % para equações

\usepackage{subfiles} % para importar arquivos de texto de outra pasta

\usepackage{longtable} % para tabelas em mais de uma página
% Necessário para incluir figuras
\usepackage{graphicx}         % pacote para importar figuras

\usepackage{times}            % pacote para usar fonte Adobe Times
% \usepackage{palatino}
% \usepackage{mathptmx}       % p/ usar fonte Adobe Times nas fórmulas

\usepackage[alf,abnt-emphasize=bf]{abntex2cite} % pacote para usar citações abnt

\usepackage{nameref}

% 
% Informações gerais
% 
\title{Some Evidence on Political Information and Exchange Coupon in Brazil}

\author{Paulsen}{Bernardo Hillesheim}

% orientador e co-orientador são opcionais (não diga isso pra eles :))
\advisor[Prof.~Dr.]{Dos Santos}{Nelson Seixas}

% a data deve ser a da defesa; se nao especificada, são gerados
% mes e ano correntes
% \date{maio}{2001}

% o local de realização do trabalho pode ser especificado (ex. para TCs)
% com o comando \location:
% \location{Itaquaquecetuba}{SP}

% itens individuais da nominata podem ser redefinidos com os comandos
% abaixo:
% \renewcommand{\nominataReit}{Prof\textsuperscript{a}.~Wrana Maria Panizzi}
% \renewcommand{\nominataReitname}{Reitora}
% \renewcommand{\nominataPRE}{Prof.~Jos{\'e} Carlos Ferraz Hennemann}
% \renewcommand{\nominataPREname}{Pr{\'o}-Reitor de Ensino}
% \renewcommand{\nominataPRAPG}{Prof\textsuperscript{a}.~Joc{\'e}lia Grazia}
% \renewcommand{\nominataPRAPGname}{Pr{\'o}-Reitora Adjunta de P{\'o}s-Gradua{\c{c}}{\~a}o}
% \renewcommand{\nominataDir}{Prof.~Philippe Olivier Alexandre Navaux}
% \renewcommand{\nominataDirname}{Diretor do Instituto de Inform{\'a}tica}
% \renewcommand{\nominataCoord}{Prof.~Carlos Alberto Heuser}
% \renewcommand{\nominataCoordname}{Coordenador do PPGC}
% \renewcommand{\nominataBibchefe}{Beatriz Regina Bastos Haro}
% \renewcommand{\nominataBibchefename}{Bibliotec{\'a}ria-chefe do Instituto de Inform{\'a}tica}
% \renewcommand{\nominataChefeINA}{Prof.~Jos{\'e} Valdeni de Lima}
% \renewcommand{\nominataChefeINAname}{Chefe do \deptINA}
% \renewcommand{\nominataChefeINT}{Prof.~Leila Ribeiro}
% \renewcommand{\nominataChefeINTname}{Chefe do \deptINT}

% A seguir são apresentados comandos específicos para alguns
% tipos de documentos.

% Relatório de Pesquisa [rp]:
% \rp{123}             % numero do rp
% \financ{CNPq, CAPES} % orgaos financiadores

% Trabalho Individual [ti]:
% \ti{123}     % numero do TI
% \ti[II]{456} % no caso de ser o segundo TI

% Monografias de Especialização [espec]:
% \espec{Redes e Sistemas Distribuídos}      % nome do curso
% \coord[Profa.~Dra.]{Weber}{Taisy da Silva} % coordenador do curso
% \dept{INA}                                 % departamento relacionado

% 
% palavras-chave
% iniciar todas com letras minúsculas, exceto no caso de abreviaturas
% 
\keyword{political information}
\keyword{financial markets}
\keyword{market efficiency}
\keyword{news}

%\settowidth{\seclen}{1.10~}

% 
% inicio do documento
% 
\begin{document}

% folha de rosto
% às vezes é necessário redefinir algum comando logo antes de produzir
% a folha de rosto:
% \renewcommand{\coordname}{Coordenadora do Curso}
\maketitle

% dedicatoria
% \clearpage
% \begin{flushright}
%     \mbox{}\vfill
%     {\sffamily\itshape
%       ``Science is the great antidote to the poison of enthusiasm and superstition..''\\}
%     --- \textsc{Adam Smith}
% \end{flushright}

% agradecimentos
\chapter*{Acknowledgments}

To my family and friends. % and to whoever contributes to the search for truth



% resumo na língua do documento
% manter 500 palavras
\begin{abstract}
    We investigate whether political news affect the exchange coupon in Brazil, in a period ranging from September 9, 2016, through April 16, 2019. Our sample of news was collected via web scrapping, which we applied on a Brazilian news portal. We use two measures for the exchange coupon, and we apply a GARCH model to estimate conditional volatility, which we filter with both a parametric and a non parametric approach. The results from the parametric analysis indicate that the exchange coupon was affected by the corruption scandal of President Temer (May 17, 2017), the annoucement of Jair Bolsonaro as candidate for the federal presidency (July 22, 2018), the first round of the presidential elections (October 8, 2018), and the announcement of changes to the Pension Reform (January 4, 2019). The results from the non parametric analysis indicate that, besides the events above, the exchange coupon was also affected by news related to the Labour Reform (March 17, 2017), to the impeachment requests of President Temer following the corruption scandal (August 17 and October 25, 2017), and to the elections for the Chamber of Deputies' presidency (January 8 and February 1st, 2019).
\end{abstract}



% lista de figuras
\listoffigures

% lista de tabelas
\listoftables

% lista de abreviaturas e siglas
% o parametro deve ser a abreviatura mais longa
% inserir ibovespa
\begin{listofabbrv}{SPMD}
    \item[PTAX] PTAX800
    \item[Selic] Special Settlement and Custody System Rate
    \item[SFN] Brazilian Financial System
    \item[CMN] National Monetary Council
    \item[BCB] Central Bank of Brazil
    \item[CoPoM] Monetary Policy Committee
    \item[CVM] Commission of Transferable Securities
    \item[BM\&F] Commodities and Futures Exchange
    \item[CETIP] Central of Custody and Financial Settlement of Private Securities
    \item[CDI] Inter-Bank Deposit Rate
    \item[ARCH] Autoregressive Conditional Heteroscedasticity
    \item[GARCH] General Autoregressive Conditional Heteroscedasticity
    \item[API] Application Programming Interface
    \item[SGS] Time Series Management System
    \item[BRL] Brazilian Real
    \item[USS] United States Dollar
    \item[VAR] Vector Autoregression
    \item[CDS] Conditional Standard Deviation
    \item[OAB] Federal Council of the Brazilian Lawyers Association
    \item[PSDB] Party of Social Democracy
    \item[PR] Liberal Party
    \item[OC1] OC1 Exchange Coupon  
    \item[DI1] DI1 Exchange Coupon
    \item[EMH] Efficient Market Hypothesis
    \item[CAPM] Capital Asset Pricing Model
    \item[TN] National Treasury
%    \item[]

\end{listofabbrv}

% sumario
\tableofcontents

% aqui comeca o texto propriamente dito

% introducao
\chapter{Introduction} \label{chapter_introduction}

    The problem we address here is whether political information affects the exchange coupon, which is the difference between the interest rate and exchange rate variation in a country. Since market efficiency in it's semi-strong form as posed by  \citet{fama1970} implies that prices reflect all information publicly available, should it prevails, exchange coupon shall not respond to political information.
  
    There is evidence on asset prices being affected by news about monetary variables (see \citet{cornell1983}), by news about the real sector (see, \citet{mcqueenroley1993}, \citet{caporaleetal2015}), and also by news about politics (\citet{marquessantos2016}). Still, in general, the semi-strong form market efficiency tests for political information in Brazil have shown supporting evidence for the hypothesis in the case of stock market returns and interest rates (\citet{marquessantos2016}). Nevertheless, local newspapers constantly refer to political news as the cause of fluctuations in prices of financial assets. Therefore, under market efficiency, these analyses are all flawed.
  
    While the work on Brazilian market efficiency cited above tests efficiency for the national investor, when using the exchange coupon we're testing efficiency for the foreign investor when investing in Brazil, as the exchange coupon is the return for dollars invested locally. Our work follows \citet{marquessantos2016} methodology: we apply web-scraping to search for news; we find the ones related to national political events by searching the headlines for keywords; we filter the exchange coupon for abnormal volatility; and finally we cross the data to find if abnormal volatility was related to political events. To find the periods with abnormal volatility for the exchange coupon, we apply a GARCH filter (\citet{bollerslev1986}), in which we search for abnormal values of the Conditional Standard Deviation series with both a parametric and a non parametric analysis.

    In Brazil, there are two measures of exchange coupon. One is the difference between the average rate of one-day inter-bank deposits (DI) and the exchange rate variation (as measured by PTAX800), while the other is the excess return of referential rate of the Special Settlement and Custody System (Selic) over exchange rate variation (PTAX800). 
  
    The paper is organized as follows: Chapter \ref{chapter_institutional} describes the Brazilian institutional environment; Chapter \ref{chapter_literature} reviews the literature on news and market efficiency; Chapter \ref{chapter_model} describes the model of market efficiency tested in this paper; Chapter \ref{chapter_methods_data} describes the data used and the methods applied; Chapter \ref{chapter_results_discussion} shows the results; and finally Chapter \ref{chapter_conclusion} shows the conclusions.

\chapter{The Institutional Environment} \label{chapter_institutional}

    Brazil's legal basis is defined in the 1988 Constitution (\citet{constituicao}). Brazil is a representative federative republic, where the government's power is divided in three branches, the Executive, Legislative and Judiciary, which are independent of each other. The country's president, chief of the Executive, is elected via direct compulsory vote by the citizens for a four-year term, and can be reelected only once. The senators and deputies, who make up the Legislative, are also elected this way, and have terms of eight and four years respectively. The constituents of the Judiciary, on the other hand, are almost all selected by pubic tender, as the ministers from the Supreme Court and Superior Justice Tribunal are indicated by the federal president.

    The Brazilian Financial System (SFN) as it is today was instituted with Law 4.595, from December 1964 (\citet{lei4595}). The National Monetary Council (CMN) was established as the major normative institution of the Financial System, while the Central Bank of Brazil (BCB) was established as the major executive institution. Since the merge of the Ministry of Finance with the Ministry of Planning, Budget and Management, in the Ministry of Economy, CMN is composed by the Minister of Economy and the Central Bank's President. CMN defines the guidelines for the budget, fiscal, monetary, credit and exchange policies, while also establishing the rules for the financial system.

    The policies which guidelines are defined by the Monetary Council are executed by the Central Bank, which goal is to enforce the norms defined by the first. The Central Bank has the monopoly of currency issue, and executes the monetary policy and the exchange policy with the buying and selling of public debt securities, which are issued by the National Treasury (TN). The president of the Central Bank is indicated by the federal president, who can replace him anytime, therefore the institution is not independent.

    Today's regimes for monetary and exchange policies started in 1999, with the establishment of the so-called Economic Tripod, which is a set of three regimes for economic policy: inflation targeting for the monetary policy, government surplus for the fiscal policy, and floating exchange rate for the exchange policy. The target for the inflation rate is defined by the Monetary Council, while the Monetary Policy Committee (CoPoM) defines the target for the short term interest rate (Selic) used for the monetary policy, and the Central Bank pursues this interest rate.

    The transferable securities market (securiries, commodities and derivatives exchanges) is disciplined and supervised by the Commission of Transferable Securities (CVM), established in Law 6.385, from December 1976 (\citet{lei6385}). Brazil has only one stock exchange, B3, which acts in all branches of the transferable securities market. B3 emerged as the fusion, in 2017, between BM\&FBOVESPA (fusion of BM\&F (Commodities and Futures Exchange) and BOVESPA (São Paulo Securities Exchange)), and CETIP (Central of Custody and Financial Settlement of Private Securities).  In Table \ref{tab:sistemanacional} the structure of the National Financial System is represented for better visualization.

    \begin{table}[H]
        \caption{National Financial System}
        \label{tab:sistemanacional}
        \centering
        \begin{tabular}{lp{2.3cm}|p{2.3cm}|p{2.3cm}|p{2.3cm}}
             & \multicolumn{2}{p{4.6cm}||}{Currency, Credit, Capital, Currency Exchange} & \multicolumn{1}{l||}{Private Insurance} & \multicolumn{1}{l}{Closed Pension} \\ \cline{2-5} 
            \multicolumn{1}{l|}{Regulating} & \multicolumn{2}{l||}{CMN} & \multicolumn{1}{p{2.3cm}||}{CNSP} & \multicolumn{1}{p{2.3cm}|}{CNPC} \\
            \multicolumn{1}{l|}{Entities} & \multicolumn{2}{l||}{(National Monetary Council)} & \multicolumn{1}{p{2.75cm}||}{(National Private Insurance Council)} & \multicolumn{1}{p{2.3cm}|}{(National Supplementary Pension Council)} \\ \hline
            \multicolumn{1}{l|}{Supervising} & \multicolumn{1}{p{2.3cm}||}{BC} & \multicolumn{1}{p{2.3cm}||}{CVM} & \multicolumn{1}{p{2.3cm}||}{Susep} & \multicolumn{1}{p{2.3cm}|}{Previc} \\
            \multicolumn{1}{l|}{Entities} & \multicolumn{1}{p{2.3cm}||}{(Central Bank of Brazil)} & \multicolumn{1}{p{2.3cm}||}{(Commission of Transferable Securities)} & \multicolumn{1}{p{2.75cm}||}{(Superintendence of Private Insurance)} & \multicolumn{1}{p{2.3cm}|}{(National Superintendence of Supplementary Pension)} \\ \hline
            \multicolumn{1}{l|}{Operators} & \multicolumn{1}{p{2.3cm}||}{Banks and Savings Banks,  Credit Cooperatives, Payment Institutions, Consortium Administrators, Brokers and Distributors, other non banking institutions} & \multicolumn{1}{p{2.3cm}||}{Stock, Commodities and Futures Exchanges} & \multicolumn{1}{p{2.75cm}||}{Insurers and Reinsurers, Open Pension Entities, Capitalization Companies} & \multicolumn{1}{p{2.3cm}|}{Closed Entities of Supplementary Pension} \\ \cline{2-5} 
        \end{tabular}
    \end{table}

\chapter{News and Market Efficiency} \label{chapter_literature}

    The Efficient Market Hypothesis (EMH), as posed by \citet{fama1970}, states that security prices "fully reflect" available information, providing "accuarate signals for resource allocation". In the model, information is divided in three subsets. For the weak form efficiency, the information set is the historical prices of the security. For the semi strong form, it is all the publicly available information, and finally, for the strong form, it is all available information, even if held private. There is massive literature on market efficiency, as the hypothesis has enormous implications for trading strategies, since it indicates the impossibility of economic profit with existing information \citet{kamal2014}. Our paper focus on semi strong form efficiency, as we work with information in the form of publicly available news.

    The evidence on the relationship between financial variables and news support that the first responds to the latter. News about the macroeconomy are shown to affect commodity prices in \citet{caporaleetal2015}, while \citet{macqueenroley1993} shows not only that the stock market responds to this subset of news, but that the response depends on the state of the economy. The use of news for trading strategies is shown to award economic profit in \citet{larsenthorsrud2017}. In \citet{moussaetal2017} it is shown that information supply has impact on prices, but the effect is more pronounced on volatility than on returns.

    \citet{caporaleetal2015} applies a VAR-GARCH model to analyze the spillovers between mean and variance of both macroeconomic news and commodity returns. The sample of returns is composed by ten commodities and the exchange rate, in a period of over 13 years. The news sample includes the worldwide coverage of four macroeconomic variables: GDP, unemployment, retail sales and durable good, which are are used in the making of two indexes, one for positive and one for negative news. The results show spillovers for all variables but gold and silver.
    In \citet{macqueenroley1993} it is analyzed if stock prices response to news vary over different stages of the business cycle. The sample of equity prices consists in the S\&P 500 Index from over 10 years, and the sample of series used to represent the stage of the economy consists of variables related to cash flows and equity discount rates. The results show that the stock market responds positively to good macroeconomic news when the economy weak, but negatively when the economy is strong.

    Textual data is used to analyze the relationship between news topics and stock returns in \citet{larsenthorsrud2017}. The news sample comes from a daily Norwegian newspaper, and the stock prices sample comes from several firms listed in the Oslo Stock Exchange. The results show that news predict daily returns, allowing for investment strategies with economic returns.
    In \citet{moussaetal2017} news headlines are used to measure information supply, while search volume from Google Trends database is used to measure information demand. The sample of stocks is 25 stocks composing the Frech stock market index CAC40 and the index itself, and the information and financial samples time range is seven years. A model is developed to test the relationship between the samples. The results indicate that information affects asset prices, but while the effect on volatility is considerable, the effect on returns is small.

    When testing for political information, the results also support that security prices are responsive. Both \citet{smales2015} and \citet{marquessantos2016} show that political uncertainty is related to market uncertainty, the first for Australia and the latter for Brazil. The Brazilian stock market is also shown to react to tax announcements in \cite{gabrielribeiro2013}.

    \citet{smales2015} uses electoral polls data to construct two measures of political uncertainty. One represents overall election uncertainty, and the other represents uncertainty about reelection, which is considered of importance as economic policies of a reelected president are well known in caparison with the policies of a newly elected president. The financial series are exchange-traded futures and options based on the main benchmark for Australia's equity markets, the S\&P/ASX 200. Market volatility is shown to increase with political uncertainty, and decrease with the probability of reelection.
    In \citet{marquessantos2016} a GARCH model is applied to daily stock returns and short term interest rates in Brazil. The sample for the first is the Bovespa Index, and the sample for the latter is composed by the Selic rate and the DI rate. Samples range from January 2014 to April 206. The results show that the the stock market only responds to political news in the case of elections, as the only period of abnormal volatility related to news happened to occur around the 2014 presidential elections' date. On the other hand, the short term interest rates do not respond to political news.
    \cite{gabrielribeiro2013} verifies if a government's annunciation of tax cuts affected stock prices of companies from the sector that would be directly affected. The referred tax is the Tax on Industrialized Products (IPI). After filtering for characteristics that would make the calculations possible, 13 stocks made to the final sample. The results indicate that the stock prices were indeed affected by the tax cut annunciation.

\chapter{The Model} \label{chapter_model}

    Market efficiency, as posed by \citet{fama1970}, is a implication of a perfect capital market (neither transaction or information costs), investor risk aversion, and two-parameter return distributions \citet{famamcbeth1973}. From the three forms, if weak form market efficiency holds, then chartist or technical analysis is useless, and if the semi-strong form holds, then fundamental analysis, founded on public information, is useless \citet{oprean2012}.

    For a brief demonstration of the model, we will consider that equilibrium prices are generated in the two parameter \citet{sharpe1964} world. Two assumptions about the investor are made: that he acts in the basis of two parameters of the distribution of returns of an asset - its expected value and standard deviation; and that he derives utility from returns and disutility from risk. The market is built upon two other assumptions: a common pure rate of interest; and homogeneity of investor expectations, as they agree on expected values, standard deviations, and correlations of the securities. In this world, the equilibrium expected return on a security is a function of it's correlation with a efficient combination of securities.
    
    Let $U$ denote the utility of an investor, $E_W$ denote the expected value of a security $W$, and $\sigma_W$ its standard deviation, the utility function of the investor is defined in Equation \ref{eqn:ut1}.
    
    \begin{equation}
        \label{eqn:ut1}
        U = f(E_W, \sigma_W)
    \end{equation}
    
    As the future value of a security is directly related to its return, let $R$ denote the return, the utility function can be defined as in Equation \ref{eqn:ut2}.

    \begin{equation}
        \label{eqn:ut2}
        U = g(E_R, \sigma_R)
    \end{equation}

    The investor likes return and dislikes risk, therefore the utility function is increasing with security's expected return and decreasing with its standard deviation, as shown in Equations \ref{eqn:der1} and \ref{eqn:der2}.
    
    \begin{equation}
        \label{eqn:der1}
        \frac{dU}{dE_R} > 0
    \end{equation}

    \begin{equation}
        \label{eqn:der2}
        \frac{dU}{d\sigma_R} < 0
    \end{equation}

    Let $i$ denote an investor among the population and $r$ denote the pure rate of interest, the assumption of a common pure rate of interest can be defined in Equation \ref{eqn:r}.

    \begin{equation}
        \label{eqn:r}
        r_i = r, \quad \forall i
    \end{equation}

    Let $a$ and $b$ denote each any security in the market, the assumption of homogeneous expectations can be defined in Equations \ref{eqn:equal_exp}, \ref{eqn:equal_std} and \ref{eqn:equal_cor}.

    \begin{equation}
        \label{eqn:equal_exp}
        E_{W_i} = E_W, \quad \forall i
    \end{equation}

    \begin{equation}
        \label{eqn:equal_std}
        \sigma_{W_i} = \sigma_W, \quad \forall i
    \end{equation}

    \begin{equation}
        \label{eqn:equal_cor}
        Cor(W_a, W_b)_i = Cor(W_a, W_b), \quad \forall i, a, b
    \end{equation}

    Let $G$ denote a efficient portfolio, the equilibrium expected return function derived from the assumptions above for any given security can be described in Equation \ref{eqn:exp_ret}.

    \begin{equation}
        \label{eqn:exp_ret}
        E_R = r + \frac{Cov(R_W, R_G)}{\sigma_{R_G}} (E_{R_G} - r)
    \end{equation}

    The model in Equation \ref{eqn:exp_ret} is called Capital Asset Pricing Model (CAPM), and shows that the equilibrium expected return for a security is a function of its risk. The assumption that expected returns adjust immediately to new information, and therefore information can't be used to gain excess returns, implies independence of successive price changes. This leads us to the random walk model. Let $r_t$ denote the security price at time $t$ and $\psi_t$ denote the information set at time $t$, the random walk model is given by Equation \ref{eqn:efficient}.

    \begin{equation}
        \label{eqn:efficient}
        f ( r_t | \psi_{t-1} ) \sim f(r_{t-1})
    \end{equation}

    As in this paper we are testing for semi-strong form efficiency, the information subset for our analysis is publicly available information. The test of efficiency can focus on a specific event like \citet{gabrielribeiro2013}, or can analyze a specific period, and test if information released during the period had effect on a security's price, like \citet{caporaleetal2015}, \citet{mcqueenroley1993}, \citet{larsenthorsrud2017}, \citet{moussaetal2017}, \citet{smales2015} and \citet{marquessantos2016}.


\chapter{Methods and Data} \label{chapter_methods_data}

    In order to test the impact of political news on the exchange coupon, we searched for political news, calculated the exchange coupon (both the OC1 and the DI1 measures), tested for abnormal volatility, and finally we crossed the periods with abnormal volatility in the exchange coupon with the correspondent political news. The information on political news was gathered with web scrapping technique, applied in the main Brazilian news portal. A filter was applied to the news sample to find the ones related to national political events. We calculated the two different measures of exchange coupon negotiated in Brazil from its components (a shared exchange rate measure and two specific interest rate measures). To find the periods with abnormal volatility, we applied a GARCH model to the exchange coupon, and then filtered it's Conditional Standard Deviation for abnormal values, with both a parametric and a non parametric analysis.

\section{Data} \label{chapter_data}

    Based on data availability, our news sample begins at September 27, 2016, the first date for which there were political news available in the scrapped website. Our exchange rate and interest rates samples begin one day earlier, to allow for the exchange coupon series to begin with the news sample. All samples end at May 16, 2019.

\subsection{Political News} \label{chapter_political_news}
    
    We gathered information on political news by applying web scrapping to a online news portal. The scrapping was made in the political section of G1, a Brazilian news portal maintained by Grupo Globo, a conglomerate based in Rio de Janeiro, that provides content from Grupo Globo's television channels, radio stations, newspapers and magazines, besides it's own content. The scrapping was executed at May 17, 2019, and resulted in a sample of 17.836 news over 896 days. The dollar market closes at 6 pm, therefore all news after this time were pushed to the subsequent day, and the news from weekdays and holidays were pushed to the closest subsequent business day.

    To find political events that could impact the Exchange Coupon, we follow \citet{bbb} as we searched our headlines sample for news that contained keywords related to uncertainty, the exchange coupon, and federal government matters. The keywords were: related to uncertainty, 'incerteza', 'mercado' and 'economia' (uncertainty, market, economy); related to the components of the exchange coupon, 'dólar', 'selic' and 'cdi' (dollar, Selic, DI) ; and related to federal government matters, 'presidente', 'presidência', 'câmara', 'senado', 'tribunal de contas da união', 'tcu', 'superior tribunal federal' and 'stf' (president, presidency, chamber (as in Chamber of Deputies), senate, Federal Accountability Office and its initials, and Supreme Court and its initials). The search resulted in a sample of 2.333 news over 855 days. The final sample is 13\% the size of the unfiltered sample in terms of number of news, therefore, we excluded 87\% of the initial sample as it was composed of political news unrelated to federal political matters.

\subsection{Exchange Coupon} \label{chapter_exchange_coupon}

    In order to obtain the series for the exchange coupon (both the OC1 and the DI1 measures), we collected a sample of the dollar exchange rate PTAX 800 (PTAX), that was used for both measures, a sample of the referential rate of the Special Settlement and Custody System (Selic) for the OC1 measure, and a sample of the of the inter-bank deposit rate (DI) for the DI1 measure. All three series were collected from the Central Bank of Brazil Time Series Management System (BACEN-SGS), through it's Application Programming Interface (API). The exchange rate is measured in BRL/USD, and the interests rates are measured in \% a day.
    
    Our samples range from September 26, 2016 to May 16, 2019. They begin shortly after the end of the impeachment process of President Dilma Rousseff, at August 31, when President Michel Temer (Ms Dilma's Vice President) took office after three months as Interim President. This happened during the largest economic crisis in Brazilian history, with the fourth trimester of 2016 being the last from a series of eleven in which the Gross Domestic Product decreased. The remaining period from our sample saw economic stagnation. Mr Temer's presidency was marked by reforms - like the Labour Reform, which successfully passed, and the Pension Reform, which was not successful -, and by a corruption scandal related to Operation Car Wash, a investigation on money laundry involving Brazilian politicians, which was very present in the media since even before his presidency. At October 2018 presidential elections were held, and Jair Bolsonaro was elected. The new president took office with he's main focus as the Pension Reform, which was proposed by the preceding president, Mr. Temer, and was (and still is by the time this is being written) awaiting for approval.
    

    Figure \ref{fig:ptax} shows the exchange rate (PTAX) series , Figure \ref{fig:selic} shows the referential rate of the Special Settlement and Custody System (Selic) series, and Figure \ref{fig:di} shows the inter-bank deposits rate (DI) series. The descriptive statistics for the series are shown in Table \ref{tab:desptaxselicdi}.

    \subfile{graphstext/ptax.txt}

    \subfile{graphstext/selic.txt}

    \subfile{graphstext/di.txt}

    \subfile{tables/desptaxselicdi.txt}

    The exchange coupon is the interest rate obtained from the difference between the accrued interest rate between the operation date and the business day preceding the due date, and the exchange rate variation observed between the business day preceding the operation date and the business day preceding the due date. We calculated both measures of the Exchange Coupon via Equation \ref{eqn:coupon}:

    \begin{equation}
        \label{eqn:coupon}
        ExchangeCoupon_t = \frac{1 + \frac{InterestRate_t}{100}}{\frac{ExchangeRate_t}{ExchangeRate_t-1}}
    \end{equation}

    Figure \ref{fig:oc} shows the OC1 exchange coupon series, Figure \ref{fig:di} shows the DI1 exchange coupon series, and Table \ref{tab:desocdi} shows the descriptive statistics for both series.

    \subfile{graphstext/oc.txt}

    \subfile{graphstext/di1.txt}

    \subfile{tables/desocdi.txt}

    Visually, the exchange coupons show a stationary form, nevertheless, we tested the series against the null hypothesis of presence of a unit root with the Augmented Dickey-Fuller (ADF) test (\citet{adf}), and against the null hypothesis of stationarity around a deterministic trend with the Kwiatkowski– Phillips–Schmidt–Shin (KPSS) test (\citet{kpss}). The results for the ADF test are shown in Table \ref{tab:ocdiadf}, and the results for the KPSS test are shown in Table \ref{tab:ocdikpss}.

    \subfile{tables/ocdiadf.txt}

    \subfile{tables/ocdikpss.txt}

    The tests results in Tables \ref{tab:ocdiadf} and \ref{tab:ocdikpss} show support for the hypothesis of stationarity of the series that the visual analysis suggested to be true. For both measures of the exchange coupon the Augmented Dickey-Fuller test rejected the null hypothesis of presence of a unit root and the Kwiatkowski– Phillips–Schmidt–Shin test accepted the null hypothesis of stationarity around a deterministic trend.
    
    \begin{comment}
        On ten of October 2015 a impeachment request for president Dilma Rousseff was handed to the Chamber Of Representatives' president, who accepted the request on December second. The opening of the impeachment process was accepted by the special impeachment committee on eleven of April 2016.
    \end{comment}

\section{Methods} \label{chapter_methods}

\subsection{Generalized Autoregressive Conditional Heteroskedastic (GARCH) Model}

    The Generalized Autoregressive Conditional Heteroskedastic (GARCH) model, introduced in \citet{bollerslev1986}, is a generalization of the Autoregressive Conditional Heteroskedastic (ARCH) model introduced in \citet{engle1982}, which models heteroscedasticity. While the most common models for time series assume a constant variance for the process (as the Vector Autoregression (VAR), popularized by \citet{var}), the ARCH process assumes an inconstant variance conditional on past variance, together with a constant unconditional variance. The GARCH process assumes inconstant variance conditional on past conditional variance as well as past variance. Both ARCH and GARCH are serial uncorrelated processes with zero mean.

    Let $y_t$ denote a real-valued discrete time process and $\psi_t$ denote the information set at time $t$, the ARCH process is given by Equation \ref{eqn:archgarch} and Equation \ref{eqn:arch} while the GARCH process is given by Equation \ref{eqn:archgarch} and Equation \ref{eqn:garch}.

    \begin{equation}
        \label{eqn:archgarch}
        y_t | \psi_{t-1} \sim N(0,h_t)
    \end{equation}

    \begin{equation}
        \label{eqn:arch}
        h_t = \alpha_0 + \displaystyle\sum_{i=1}^{q} \alpha_i y_{t-i}^2
    \end{equation}

    \begin{equation}
        \label{eqn:garch}
        h_t = \alpha_0 + \displaystyle\sum_{i=1}^{q} \alpha_i y_{t-i}^2 + \displaystyle\sum_{i=1}^{p} \beta_i h_{t-i}
    \end{equation}

    where

    $$q > 0, \quad   p \geq 0$$

    $$\alpha_0 > 0, \quad  \alpha_i \geq 0, \quad  i=1,...,q$$

    $$\beta_i \geq 0, \quad  i=1,...,p$$

    As we can see above, if $p = 0$ than it becomes a ARCH process. If $q = 0$ than the process is white noise. We will use $h_t$, the Conditional Standard Deviation, as a estimate for the exchange coupon's standard deviation.

\subsection{Estimation} \label{chapter_estimation}

    We estimate a GARCH model for both measures of the exchange coupon. The model estimated will be a GARCH(1,1), therefore we must visually check if one lag for both the auto regressive and the moving average parts of the model is appropriate. We do so by visually inspecting the auto-correlation and partial auto-correlation functions. The graphs for the auto-correlation function and for the partial auto-correlation function are given in Figure \ref{fig:ocacf} and Figure \ref{fig:ocpacf}, respectively, for the OC1 exchange coupon, and in Figure \ref{fig:diacf} and Figure \ref{fig:dipacf} for the DI1 exchange coupon.

    \subfile{graphstext/ocacf.txt}

    \subfile{graphstext/ocpacf.txt}

    \subfile{graphstext/diacf.txt}

    \subfile{graphstext/dipacf.txt}

    The visual inspection shows support for the use of GARCH(1,1). After the fitting of the model, the residuals must behave like white noise, that is, the mean must be constant and equal to zero, and there must be no auto-correlation in the series. First, we inspect this behavior visually with the residuals graph and auto-correlation function. The graph for the residuals is shown in Figure \ref{fig:ocres} for the OC1 measure, and in Figure \ref{fig:dires} for the DI1 measure. The graph for the auto-correlation function is shown in Figure \ref{fig:ocresacf} for the OC1 measure and in Figure \ref{fig:diresacf} for the DI1 measure.

    \subfile{graphstext/ocres.txt}

    \subfile{graphstext/ocresacf.txt}

    \subfile{graphstext/dires.txt}

    \subfile{graphstext/diresacf.txt}

    The visual inspection shows white noise behaviour. To certify this behaviour, we test the null hypothesis that the residuals are independently distributed with the Ljung-Box test (\citet{boxpierce} and \citet{ljungbox}), and the null hypothesis that the residuals sample comes from a normal distributed population with the Shapiro-Wilk test (\citet{shapirowilk}). The results for the tests, for both measures of the exchange coupon, are shown in Table \ref{tab:reswhite}.

    \subfile{tables/reswhite.txt}

    The results shown in Table \ref{tab:reswhite} support the hypothesis that the residuals behave as white noise. The Ljung-Box tests cannot reject the null hypothesis that the residuals are independently distributed, while the Shapiro-Wilk tests accepted the null hypothesis that the residuals sample comes from a normal distributed population.

\subsection{Volatility Estimate}

    As measure of volatility we use the Conditional Standard Deviation (CSD) series extracted from the GARCH model. The CSD series are shown in Figure \ref{fig:occsd} for the OC1 measure, and in Figure \ref{fig:dicsd} for the DI1 measure. The descriptive statistics for the series are shown in Table \ref{tab:descsd}

    \subfile{graphstext/occsd.txt}

    \subfile{graphstext/dicsd.txt}

    \subfile{tables/descsd.txt}

    We consider abnormal volatility every value outside the 95\% confidence interval. We used both a  and a non parametric analysis to filter for abnormal volatility. In the former, we assume a two-parameter distribution for the population, while in the latter we do not make this assumption.

\subsection{Parametric}

    In the parametric analysis, we assume a two parameter distribution for the Conditional Standard Deviation series when computing the 95\% confidence interval that will be used to filter the series for abnormal volatility. The upper and lower limits are defined in Equations \ref{eqn:upp} and \ref{eqn:lop}: 

    $$\bar{X} = \frac{1}{n} \displaystyle\sum_{i=0}^{n} CSD_t$$

    $$\sigma^2 = \frac{1}{n-1} \displaystyle\sum_{i=0}^{n} (CSD_t - \bar{X})^2$$

    $$\sigma = \sqrt{\sigma^2}$$

    \begin{equation}
        \label{eqn:upp}
        UpperLimit_t = \bar{X} + 1.96 * \sigma
    \end{equation}

    \begin{equation}
        \label{eqn:lop}
        LowerLimit_t = \bar{X} - 1.96 * \sigma
    \end{equation}

    We test the the null hypothesis that the Conditional Standard Deviation samples come from a normal distributed population with the Shapiro-Wilk test, which results are shown in Table \ref{tab:csdshapiro}.

    \subfile{tables/csdshapiro.txt}

    The test results shown in Table \ref{tab:csdshapiro} support the null hypothesis.

    The lower and upper limits are shown in Table \ref{tab:limpar} for both measures of the Exchange Coupon. Figure \ref{fig:oclimpar} and Figure \ref{fig:dilimpar} show the upper and lower limits altogether with the CDS series, for the OC1 and DI1 exchange coupons respectively.

    \subfile{tables/limpar.txt}

    \subfile{graphstext/oclimpar.txt}

    \subfile{graphstext/dilimpar.txt}

    Table \ref{tab:ocparout} and Table \ref{tab:diparout} show the details of each day with abnormal volatility, that is, the days in which the Conditional Standard Deviation was outside the limits of the 95\% confidence interval, for the OC1 exchange coupon and for the DI1 exchange coupon respectively.

    \subfile{tables/ocparout.txt}

    \subfile{tables/diparout.txt}

    As we see in the tables above, the results are basically the same for both measures of the exchange coupon. There were 18 days of abnormal volatility in our 685 days sample, of which 4 are from 2016, 5 are from 2017, 8 are from 2018, and 1 is from 2019. Of these 18 days, in 10 the exchange coupons show positive values.


\subsection{Non Parametric}

    While in the parametric analysis we assumed a normal distribution for the series, in the non-parametric analysis we do not make this assumption. We calculated the mean and standard deviation for each day with a 63 days window (22 days before and 22 days after), which corresponds to three months of data. For the days at the beginning and end of the sample for which there were not 22 days before, or after, available, the calculation was made with the available days. The definition of the 63 days windows was made based on the frequency of CoPon meetings (one for every one and a half month), in which the goal for the Selic is defined, affecting directly both the Selic and the DI. The window is twice the period between meetings, therefore a quarter of year.

    We define the limits of the non-parametric analysis in Equations \ref{eqn:upnp} and \ref{eqn:lonp}.

    $$\bar{X}_t = \frac{1}{63} \displaystyle\sum_{i=t-22}^{t+22} CSD_t$$

    $$\sigma^2_t = \frac{1}{62} \displaystyle\sum_{i=t-22}^{t+22} (CSD_t - \bar{X}_t)^2$$

    $$\sigma_t = \sqrt{\sigma^2t}$$

    \begin{equation}
        \label{eqn:upnp}
        UpperLimit_t = \bar{X}_t + 1.96 * \sigma_t
    \end{equation}

    \begin{equation}
        \label{eqn:lonp}
        UpperLimit_t = \bar{X}_t - 1.96 * \sigma_t
    \end{equation}

    The lower and upper limits are shown in Table \ref{tab:limnon} for both measures of the Exchange Coupon. Figure \ref{fig:oclimnon} and Figure \ref{fig:dilimnon} show the upper and lower limits altogether with the CDS series, for the OC1 and DI1 exchange coupons respectively.

    \subfile{tables/limnon.txt}

    \subfile{graphstext/oclimnon.txt}

    \subfile{graphstext/dilimnon.txt}

    Table \ref{tab:ocnonout} and Table \ref{tab:dinonout} show the details of each day with abnormal volatility, that is, the days in which the conditional standard deviation was outside the limits of the 95\% confidence interval, for the OC1 exchange coupon and for the DI1 exchange coupon respectively.

    \subfile{tables/ocnonout.txt}

    \subfile{tables/dinonout.txt}

    As we see in the tables above, the results are basically the same for both measures of the exchange coupon. There were 28 days of abnormal volatility in our 685 days sample, of which 4 are from 2016, 8 are from 2017, 10 are from 2018, and 6 are from 2019. Of these 28 days, in 16 the exchange coupons show positive values.

\chapter{Results and Discussion} \label{chapter_results_discussion}

    After having filtered the exchange coupon for abnormal volatility, and the news for national political events, we match the two samples to find what political events happened to occur in the days of abnormal volatility. With this information we can analyze the impact of political news in the exchange coupon.

    The days of abnormal volatility for both measures of the exchange coupon were the same in each approach (parametric and non parametric), therefore we will talk about the results without referencing a specific measure.

\section{Parametric}

    The political news for each day of abnormal volatility for the parametric analysis are shown in Table \ref{tab:parnews}. There are a total of 69 news in 14 days of abnormal volatility.

    \subfile{tables/parnews.txt}

    The parametric analysis shows 5 periods of abnormal volatility. The first period started at April 19, 2017 with the corruption scandal of President Michel Temer, and lasted until April 25 - therefore, a whole week. The scandal begun with the disclosure by the Supreme Court of a recording of a conversation between Mr. Temer and the businessman Joesley Batista - owner of JBS, which was, and still is by the time this is written, the largest meat processing company in the world -, who delivered the recording to the authorities as part of a plea bargain. The date in which the recording was revealed was commonly dubbed as 'Joesley Day', as it was a day of of high volatility both in the stock and in the dollar markets. In fact, the Conditional Standard Deviation for both measures of exchange coupon was of 0.023, the mean plus 13,72 times the standard deviation.

    The second period was between June 11 and 13, 2019, when news dealt with the visit of Paraguay's newly elected president Marido Abdo Benítez to president Temer, with a encounter of Petrobra's President with the Legislative and Judiciary, and with the change of judges from Superior Justice Tribunal in a process from Operation Car Wash, among other topics. The third period lasted only a single day, the 23th of July, 2018. In this date both Jair Bolsonaro, from Liberal Social Party, and Guilherme Boulos, from Socialism and Liberty Party, were confirmed as condidates for the presidential elections that would occur in October.

    The second last period of abnormal volatility begun 6 days before the first round of the 2018 presidential elections, and ended two days after the first round. At October 4th, when the period of abnormal volatility started, a voter intent survey was released which result was, in the case of a second round between Jair Bolsonaro and Fernanando Haddad, of 43\% of votes for the first and 42\% for the latter.

    The last period was the January 4th, at the time, Bolsonaro was talking about trying to pass a new Pension Reform rather than working on the approval of the reform from the previous president, Temer.

    After knowing the political news that match with the abnormal volatility days, we are able to hypothesize on which political news actually had an effect in the exchange coupon. We considered that mainly matters related to changes in the presidency affected the financial variable, followed by matters about the Pension Reform. Both the arise of the possibility of a impeachment of president Temer caused by a corruption scandal, and the presidential election with two very opposing leading candidates, brought abnormal volatility. The annoucement of a new Pension Reform also had this effect. These periods are shown in Table \ref{tab:respar}.

    \begin{table}[H]
        \caption{Periods of Abnormal Volatility related to Political News, by Parametric Analysis}
        \label{tab:respar}
        \centering
        \begin{tabular}{| c | c |}
            \hline
            Period & News Topic \\
            \hline \hline
            17/05/19 - 17/05/25 & President Temer's corruption scandal \\
            \hline
            18/07/23 & Announcement of Mr. Bolsonaro's candidacy \\
            \hline
            18/10/04 - 18/10/10 & First Round of Presidential Elections \\
            \hline
            19/01/04 & Announcement of new Pension Reform \\
            \hline
        \end{tabular}
    \end{table}

\section{Non Parametric}

    The political news for each day of abnormal volatility, for the non parametric analysis, are shown in Table \ref{tab:nonnews}. There are a total of 15 news in 25 days of abnormal volatility.

    \subfile{tables/nonnews.txt}

    The 25 days of abnormal volatility are divided in 13 periods. At March 17, 2017, there was a news about the voting of the Labour Reform. The three days of abnormal volatility between April 19 and 23, 2017, saw the beginning of the corruption scandal of president Temer, as mentioned in the previous section. Abnormal volatility returned at August 17, when the Federal Council of the Brazilian Lawyers Association (OAB) triggered the Supreme Court to make the Chamber of Representatives' President analyze the impeachment requests he had received (there were 25 at the time). The impeachment matter correlated with abnormal volatility for the last time at October 25, when the Chamber of Representatives voted for the corruption denunciation against president Temer not to be sent to the Supreme Court, where it would be judged. At December 1st, is was disclosed that the president of the Party of the Republic (PR) used a apartment of the Chamber of Deputies while fugitive (he was convicted for corruption).

    Abnormal volatility for the year of 2018 started in a period ranging from January 26 to January 29, where news about corruption being more decisive than economics in the presidential elections were published. Between June 11 and June 13, there were 12 news, from which the main ones were listed in the previous sub chapter. At July 23, both Jair Bolsonaro, from Liberal Social Party, and Guilherme Boulos, from Socialism and Liberty Party, were confirmed as condidates for the presidential elections that would occur in October. As in the parametric analysis, the days surrounding the first round of the presidential elections (between October 4 and 10) were of abnormal volatility.

    The forth day of Bolsonaro's presidency was the first day of his presidency to have abnormal volatility. At the time, Bolsonaro was talking about trying to pass a new Pension Reform rather than working on the approval of the reform from the previous president, Temer. From January 7 to January 8, both Brazilian Party of Social Democracy (PSDB), Liberal Party (PR) and party Podemos (called National Labour Party until 2016) announced support for the reelecion of Rodrigo Maia for the Chamber of Deputies' presidency. The reelection occurred at February 1st, the date with the most political news - 34- and correlated with abnormal volatility. At March 25 Chamber of Representative's president, Rodrigo Maia, said president Bolsonaro would lead the political articulation for the Pension Reform, and three days later the rapporteur for the Reform was chosen. The last period of abnormal volatility for our sample ranged from April 1st to 2nd, when the Plateau Palace, house of president Bolsonaro, share a video negating the denomination 'coup' for the military movement of 1964 which changed the government.

    The results from the non parametric analysis add to the results of the parametric analysis, as they show that not only matters related to changes in the presidency and the Pension Reform affected the exchange coupon, but matters related to the presidency of the Chamber of Deputies and the Labour reform also did - the remaining news topics from our sample were considered of minor significance. Even in the matters related to the presidency, more political events affected the exchange coupon. The possibility of impeachment of president Michel Temer impacted the exchange coupon not only in May 2017 (as in the parametric analysis), but in August and October as well.In the first, OAB triggered the Supreme Court to make the Chamber of Representatives' President analyze the impeachment requests he had received, while in the latter the denunciation against the president was rejected by the Chamber of Deputies. The elections for the presidency of the Chamber of Deputies, that took place in February 2019, also brought abnormal volatility. The announcement of the voting of the Labour Reform also had effect. The last political event that was related the abnormal volatility in the exchange coupon was the announcement of a new Pension Reform. The periods of abnormal volatility related to political news, by the parametric analysis, are shown in Table \ref{tab:resnon}.

    \begin{table}[H]
        \caption{Periods of Abnormal Volatility related to Political News, by Non Parametric Analysis}
        \label{tab:resnon}
        \centering
        \begin{tabular}{| c | c |}
            \hline
            Period & News Topic \\
            \hline \hline
            17/03/17 & Voting of the Labour Reform \\
            \hline
            19/05/17 - 23/05/17 & President Temer's corruption scandal \\
            \hline
            17/08/17 & OAB's triggering of the Supreme Court \\
            \hline
            17/10/25 & Rejection of the denunciation against president Temer \\
            \hline
            18/07/23 & Announcement of Mr. Bolsonaro's candidacy \\
            \hline
            18/10/04 - 18/10/10 & First Round of Presidential elections \\
            \hline
            19/01/08 & PSDB's support for Rodrigo Maia's candidacy \\
            \hline
            19/02/01 & Election for Chamber of Deputies' presidency \\
            \hline
            19/03/28 & Announcement of a new Pension Reform \\
            \hline
        \end{tabular}
    \end{table}

\chapter{Conclusion} \label{chapter_conclusion}

    In order to test the market for semi-strong form efficiency, we analyzed the impact of political news on the country's exchange coupon. We cross referenced the days with abnormal returns for the exchange coupon with the days with political news. Ir order to achieve that, we filtered the news collected via web scrapping for national political events, and applied a GARCH filter in the exchange coupon to filter for abnormal returns. We performed both a parametric and a non parametric analysis.

    The results, from both the parametric and non parametric analysis, show support for the semi-strong form efficient market hypothesis. From 855 days with political news, only a small fraction had effect over the exchange coupon (17 days in the case of the parametric analysis, and 47 days in the case of the non parametric analysis).

    The parametric and non parametric analysis differ in their results, but both indicate that the exchange coupon was affected by political news about changes in the federal presidency and the Pension Reform. Changes in the presidency means both the 2018 election, with its first round and the announcement of later-to-become president Jair Bolsonaro as candidate, and the corruption scandal with President Michel Temer at May 2017, which was followed by various impeachment requests. The Pension Reform had its effect with the announcement by President Bolsonaro of changes in the one proposed by Michel Temer.

    While the parametric analysis indicates that the relationship between political news and the exchange coupon is limited to the one mentioned above, the non parametric analysis indicates that this relationship goes further, as political news of less (but in no way small) significance also affect the exchange coupon. In the case of the corruption scandal with President Temer, abnormal volatility was grought also by the triggering of the Supreme Court, by the Federal Council of the Brazilian Lawyers Association (OAB), to make the Chamber of Representatives' President analyze the impeachment requests he had received, and by the voting by the Chamber of Deputies for the corruption denunciation not to be sent to the Supreme Court. In the non parametric analysis, not only the Pension Reform affected the exchange coupon but also the Labour Reform, when the Chamber of Deputies' president announced its voting. Other than the elections for the federal presidency, the elections for the Chamber of Deputies' presidency also were shown to affect the coupon, and as in the federal, both announcement of support for Rodrigo Maria's candidacy and the election itself caused abnormal volatility.

    These results follow \cite{smales2015} and \cite{marquessantos2016}, as they both show that political uncertainty, as with who will win the elections, causes market uncertainty. The results from \cite{marquessantos2016} don't show, however, that other matters related to presidency cause market uncertainty, even though the sample used contained the period with the impeachment process of President Dilma Rouseff. This difference in results may arrive from differences between the foreign and national investors, with the first being more sensitive to political information.



















    
\chapter{Appendix - Codes} \label{chapter_appendix}

\section{Main}

    \subfile{codes/main.tex}

\section{Modules}

\subsection{Spider}

    \subfile{codes/spider.tex}

\subsection{Import and Process}

    \subfile{codes/import_process.tex}

\subsection{Output of Graphs}

    \subfile{codes/process_output_graphs.tex}

\subsection{Output of Tables}

    \subfile{codes/process_output_tables.tex}

\bibliographystyle{abntex2-alf}
\bibliography{ref_final} \label{chapter_bibliography}

\end{document}
