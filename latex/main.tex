\documentclass[a4paper]{article}

%% Language and font encodings
\usepackage[english]{babel}
\usepackage[utf8]{inputenc}
\usepackage[T1]{fontenc}
\usepackage[colorinlistoftodos]{todonotes}

%% Sets page size and margins
\usepackage[a4paper,top=3cm,bottom=2cm,left=3cm,right=3cm,marginparwidth=1.75cm]{geometry}

%% Useful packages
\usepackage{amsmath}
\usepackage{mathtools}

\usepackage{graphicx}

\usepackage{subfiles}
\usepackage{longtable}



\usepackage{float}
\usepackage[colorinlistoftodos]{todonotes}
\usepackage[colorlinks=true, allcolors=blue]{hyperref}
\usepackage{verbatim}
\usepackage[natbibapa]{apacite}
\providecommand{\keywords}[1]{\textbf{\textit{Keywords}} #1}
\providecommand{\jel}[2]{\textbf{\textit{JEL Classification}} #2}


\title{Some Evidence on Political Information and Exchange Rate Coupon in Brazil}
\author{Nelson Seixas dos Santos\\Faculdade de Ciências Econômicas\\Universidade Federal do Rio Grande do Sul\\Bernardo Hillesheim Paulsen\\Faculdade de Ciências Econômicas\\Universidade Federal do Rio Grande do Sul}



\begin{document}

\maketitle
\clearpage
\tableofcontents
\clearpage
\listoffigures
\clearpage
\listoftables
\clearpage

\begin{abstract}

\end{abstract}


\section{Introduction}

The problem we address here is whether political information affects the exchange coupon, which is the difference between the interest rate and exchange rate variation in a country. Since market efficiency in it's semi-strong form as posed by  \citet{fama1970} implies that prices reflect all information publicly available, should it prevails, exchange coupon shall not respond to political information. \todo{que}

There is evidence on asset prices being affected by news about monetary variables (see \citet{cornell1983}), by news about the real sector (see, \citet{mcqueenroley1993}, \citet{caporaleetal2015}), and also by news about politics \citet{marquessantos2016}. In general, the semi-strong form market efficiency tests for political information in Brazil have shown supporting evidence for the hypothesis in the case of stock market returns and interest rate \citet{marquessantos2016}, and in the case of term structure of interest rate \citet{santos2018}. Nevertheless, local newspapers constantly refer to political news as the cause of fluctuations in prices of financial assets. Therefore, under market efficiency, these analyses are all fundamentally flawed.

While the work on Brazilian market efficiency cited above tests for efficiency for the national investor, when using the exchange coupon we're testing for efficiency for the foreign investor when investing in Brazil, as the exchange coupon is the return for dollars invested in Brazil. Our work follows \citet{marquessantos2016} methodology: we use web-scraping to search for political news; we filter the news for national political events and the exchange coupon for abnormal volatility; and finally we cross the data to find national political events on the days of abnormal volatility. To filter the news sample we follow XXXXXXXX, as we search for the news that contain keywords related to national political events. To find the periods with abnormal volatility for the exchange coupon, we apply a GARCH filter (\citet{bollerslev1986}) to the series, in which we search for abnormal values with both a parametric and a non parametric analysis.

In Brazil, there are two measures of exchange coupon. The first one is the (daily) difference between the average rate of one-day interbank deposits of DI1 and the (dollar) exchange rate variation (as measured by PTAX800) and the other one is the excess return of referential rate of the Special Settlement and Custody System (Selic) over (dollar) exchange rate variation (PTAX800).


\section{The Institutional Environment}


Brazil's legal basis is defined in the 1988 Constitution \citet{constituicao}. Brazil is a representative federative republic, where the government's power is divided in three branches, the Executive, Legislative and Judiciary, which are independent of each other. The country's president, chief of the Legislative, is elected via direct compulsory vote by the citizens for a four-year term, and can be reelected only once. The senators and deputies, who make up the Legislative, are also elected this way, but have terms of eight and four years respectively. The constituents of the Judiciary, on the other hand, are almost all selected by pubic tender, as the supreme ministers are indicated by the country's president.

The Brazilian Financial System (Sistema Financeiro Nacional - SFN) as it is today was instituted with \citet{lei4595}, from December 1964. The National Monetary Council (Conselho Monetário nacional - CMN) was established as the major normative institution of the Financial System, while the Central Bank of Brazil (Banco Central do Brasil - BCB or BaCen) was established as the major executive institution. Since the merge of the Ministry of Finance with the Ministry of Planning, Budget and Management, in the Ministry of Economy, the Monetary Council is composed by the Minister of Economy and the Central Bank's President. The Monetary Council defines the guidelines for the budget, fiscal, monetary, credit and exchange policies, while also establishing the rules for the financial system.

The policies which guidelines are defined by the Monetary Council are executed by the Central Bank, which goal is to enforce the norms defined by the first. The Central Bank has the monopoly of currency issue, and executes the monetary policy and the exchange policy with the buying and selling of public debt securities, which are issued by the National Treasury. The president of the Central Bank is indicated by the country's president, and can be replaced anytime, therefore the institution is not independent.

Today's regimes of monetary and exchange policies started in 1999 \todo{referencia}, with the establishment of the so-called Economic Tripod, which is a set of three regimes for economic policy: inflation targeting for the monetary policy, government surplus for the fiscal policy, and floating exchange rate for the exchange policy. The target for inflation rate is defined by Monetary Council (Conselho Monetário), the Monetary Policy Committee (Comissão de Política Monetária - CoPoM) defines the target for the short term interest rate (Selic) used for the monetary policy, and the Central Bank pursues this interest rate.

The transferable securities market (securiries, commodities and derivatives exchanges) is disciplined and supervised by the Commission of Transferable Securities (Comissão de Valore Mobiliários - CVM), established in \citet{lei6385} from December 1976. Brazil has only one stock exchange, B3, which acts in all branches of the transferable securities market. B3 emerged as the fusion, in 2017, between BM\&FBOVESPA (fusion of BM\&F and BOVESPA), and CETIP.  In Table \ref{tab:sistemanacional} the structure of the National Financial System is represented for better visualization.

\begin{table}[H]
\caption{National Financial System}
\label{tab:sistemanacional}
\centering
\begin{tabular}{lp{2.75cm}|p{2.75cm}|p{2.75cm}|c}
 & \multicolumn{2}{p{5.5cm}||}{Currency, Credit, Capital, Currency Exchange} & \multicolumn{1}{l||}{Private Insurance} & \multicolumn{1}{l}{Closed Pension} \\ \cline{2-5} 
\multicolumn{1}{l|}{Regulating} & \multicolumn{2}{c||}{CMN} & \multicolumn{1}{p{2.75cm}||}{CNSP} & \multicolumn{1}{p{2.75cm}|}{CNPC} \\
\multicolumn{1}{l|}{Entities} & \multicolumn{2}{c||}{(National Monetary Council)} & \multicolumn{1}{p{2.75cm}||}{(National Private Insurance Council)} & \multicolumn{1}{p{2.75cm}|}{(National Supplementary Pension Council)} \\ \hline
\multicolumn{1}{l|}{Supervising} & \multicolumn{1}{p{2.75cm}||}{BC} & \multicolumn{1}{p{2.75cm}||}{CVM} & \multicolumn{1}{p{2.75cm}||}{Susep} & \multicolumn{1}{p{2.75cm}|}{Previc} \\
\multicolumn{1}{l|}{Entities} & \multicolumn{1}{p{2.75cm}||}{(Central Bank of Brazil)} & \multicolumn{1}{p{2.75cm}||}{(Commission of Transferable Securities)} & \multicolumn{1}{p{2.75cm}||}{(Superintendence of Private Insurance)} & \multicolumn{1}{p{2.75cm}|}{(National Superintendence of Supplementary Pension)} \\ \hline
\multicolumn{1}{l|}{Operators} & \multicolumn{1}{p{2.75cm}||}{Banks and Savings Banks,  Credit Cooperatives, Payment Institutions, Consortium Administrators, Brokers and Distributors, other non banking institutions} & \multicolumn{1}{p{2.75cm}||}{Stock, Commodities and Futures Exchanges} & \multicolumn{1}{p{2.75cm}||}{Insurers and Reinsurers, Open Pension Entities, Capitalization Companies} & \multicolumn{1}{p{2.75cm}|}{Closed Entities of Supplementary Pension} \\ \cline{2-5} 
\end{tabular}
\end{table}

\section{The model}

Market efficiency, as posed by \citet{fama1970}, suggest that prices reflect available information, providing accurate signals for resource allocation. Prices should reflect information as it reflects the forecasts of future prices, which should be revised with the arrival of new information. Efficiency is a implication of a perfect capital market, investor risk aversion, and two-parameter return distributions \citet{famamcbeth1973}.

Tree information subsets define tree forms of market efficiency: for the weak form market efficiency the information set reflected in the price is the historical prices; for the semi-strong form the information set is all publicly available information (including historical prices); and for the strong form the information set is all available information, even if held private. If weak form market efficiency holds, then chartist or technical analysis is useless, and if the semi-strong form holds, then fundamental analysis, founded on public information, is useless \citet{oprean2012}.

Testing of semi-strong form market efficiency is done by testing for the existence of abnormal volatility around the moment when new information was released. The test for abnormal returns can focus on a specific event like \citet{gabrielribeiro2013}, or can focus on a asset during a specific period, testing if the asset was affected by any information of a specific form or source during the period, like \citet{larsenthorsrud2017} and \citet{caporaleetal2015}.

\begin{comment}

\todo{Buscar outras citações, lembre do Mandelbrot e do Fama 1963 Journal of Busines})

As a consequence, \citet{pearce1987} states If the market is efficient there is no economic profit to be done with publicly available information, and investors should expect normal returns for the risk they bear, independent of the (publicly available) information they have access to.

Recent work on the effect of news on financial prices report a significant impact of those news on commondities and stock (\citet{larsenthorsrud2017} \citet{caporaleetal2015}).

We use the supply of information instead of demand for information because, if information is known by few people, then it's private, therefore using demand for information as a proxy for access to information would be testing the market for strong form efficiency.

Tests of market efficiency vary depending on the form of efficiency being tested. Tests for the weak form consist in testing if the historical prices follow a random walk (thus if they have a unit root). This is because, if current prices reflect historical prices, then there is no economic profit to be done with historical prices, so the best forecast for the price at the next period is always the current price, what defines the price as a random walk. Tests of weak form market efficiency include \citet{dongetal2013}, \citet{kamal2014}, and \citet{narayanetal2016}, which all reject weak form market efficiency.

The rationale behind this method was mentioned in the introduction. The use of supply of (rather than the demand for) information when testing is because, if information is know by few, then it's private, therefore demand for information, as a proxy for access to information, has use for testing strong form efficiency.

Strong form market efficiency is tested by testing if information risk is priced. This is because, if information leads to greater returns, then there is a risk of information, and assets with more information have greater returns. Therefore, if information risk is priced, the market is inefficient. The results of tests using accruals quality as a proxy for information risk  are mixed, with evidence of information risk being priced \citet{francisetal2005}, \citet{grayetal2009}, \citet{kimqi2010}, \citet{safdaryan2017} and not being priced \citet{coreetal2007}, \citet{mousellietal2013}. When PIN (probability of information-based trading) variable is used instead, the result is information risk being priced \citet{easleyetal2002}. However, \citet{duarteyoung2009}  showed PIN being priced for its illiquidity component, and not for its information asymmetry component \citet{safdaryan2017}.

\end{comment}

\section{Methods and Data}

In order to test the impact of political news on the Exchange Coupon, we searched for political news, calculated the Exchange Coupon series (for both the OC1 and the DI1 measures), tested for abnormal volatility in the series, and finally we crossed the periods with abnormal volatility in the Exchange Coupon with the correspondent political news. The information on political news was gathered with web scrapping technique, applied in some of the main Brazilian news sources. We calculated the two different measures of Exchange Coupon negotiated in Brazil from the series of its components (a shared exchange rate measure series and two specific interest rate measures). To find the periods with abnormal volatility, we applied a GARCH to the Exchange Coupon series, and then filtered it's Conditional Standard Deviation series for abnormal values, with both a parametric and a non parametric analysis.


\subsection{Data}

Based on data availability, our sample begins on September 9, 2016, the last date for which there were political news available in the scrapped website.

\subsubsection{Political News}

We gathered information on political news by applying web scrapping to a online newspaper. The scrapping was made in the political section of the Brazilian newspaper G1, a news portal maintained by Grupo Globo, a conglomerate based in Rio de Janeiro (county with the second largest GDP in Brazil), that provides content from Grupo Globo's television channels, radio stations, newspapers and magazines, besides it's own content. The scrapping was done at May 17, 2019, and resulted in a sample of 17.836 news over 896 days.

\todo{over XXX days, resulting in a mean of XX sampler per day}

To find political events that could impact the Exchange Coupon, \todo{we follow REFERENCIA} we searched our sample for news that contained keywords related to uncertainty, the exchange coupon, and federal government matters. The keywords were: related to uncertainty, 'incerteza', 'mercado' and 'economia' (uncertainty, market, economy); related to the components of the exchange coupon, 'dólar', 'selic' and 'cdi' (dollar, inter-bank deposit certificate, Special Settlement and Custody System) ; and related to federal government matters, 'presidente', 'presidência', 'câmara', 'senado', 'tribunal de contas da união', 'tcu', 'superior tribunal federal' and 'stf' (president, presidency, chamber (as in chamber of deputies), senate, federal accountability office and its initials, and supreme court and its initials). The search resulted in a sample of 2.333 news over 855 days. The final sample is 13\% the size of the unfiltered sample in terms of number of news, therefore, we excluded 87\% of the initial sample as it was composed of political news unrelated to matters that could impact the exchange coupon.

\todo{colocar as notícias do fim de semana no dia de segunda-feira}

\subsubsection{Exchange Coupon}

In order to obtain the series for the Exchange Coupon (both the OC1 and the DI1 measures), we collected a sample of the dollar exchange rate (PTAX 800), that was used for both measures, a sample of the referential rate of the Special Settlement and Custody System (Selic) for the OC1 measure, and a sample of the of the Inter-bank Deposit rate (CDI) for the DI1 measure. All three series were collected from the Central Bank of Brazil Time Series Management System (BACEN-SGS), through it's Application Programming Interface (API). The exchange rate is measured in BRL/USS, and the interests rates are measured in \% a day.

Figure \ref{fig:ptax} shows the exchange rate series , Figure \ref{fig:selic} shows the referential rate of the Special Settlement and Custody System series, and Figure \ref{fig:di} shows the Interbank Deposits rate series. The descriptive statistics for the series are shown in Table \ref{tab:desptaxselicdi}.

\subfile{graphstext/ptax.txt}

\subfile{graphstext/selic.txt}

\subfile{graphstext/di.txt}

\subfile{tables/desptaxselicdi.txt}

\todo{comentar sobre o que aconteceu no período no Brasil, como parte da descrição dos dados}


The Exchange Coupon is the interest rate obtained from the difference between the accrued interest rate between the operation date and the business day preceding the due date, and the exchange rate variation observed between the business day preceding the operation date and the business day preceding the due date. We calculated both measures of the Exchange Coupon via Equation \ref{eqn:coupon}:

\begin{equation}
\label{eqn:coupon}
ExchangeCoupon_t = \frac{1 + \frac{InterestRate_t}{100}}{\frac{ExchangeRate_t}{ExchangeRate_t-1}}
\end{equation}

Figure \ref{fig:oc} shows the OC1 Exchange Coupon series, Figure \ref{fig:di} shows the DI1 Exchange Coupon series, and Table \ref{tab:desocdi} shows the descriptive statistics for the both series.

\subfile{graphstext/oc.txt}

\subfile{graphstext/di.txt}

\subfile{tables/desocdi.txt}

Visually, the Exchange Coupon series show a stationary form, nevertheless, we tested the series against the null hypothesis of presence of a unit root with the Augmented Dickey-Fuller (ADF) test (\citet{adf}), and against the null hypothesis of stationarity around a deterministic trend with the Kwiatkowski– Phillips–Schmidt–Shin (KPSS) test (\citet{kpss}). The results for the ADF test are shown in Table \ref{tab:ocdiadf}, and the results for the KPSS test are shown in Table \ref{tab:ocdikpss}.

\todo{talvez comentar sobre o processo de testes, tanto anterior à modelagem como de teste da modelagem, a partir do livro da metodologia box jenkins}

\subfile{tables/ocdiadf.txt}

\subfile{tables/ocdikpss.txt}

The tests results in Tables \ref{tab:couponadf} and \ref{tab:couponkpss} show support for the hypothesis of stationarity of the series that the visual analysis suggested to be true. For both measures of the Exchange Coupon the Augmented Dickey-Fuller test rejected the null hypothesis of presence of a unit root and the Kwiatkowski– Phillips–Schmidt–Shin test accepted the null hypothesis of stationarity around a deterministic trend.


\begin{comment}

On ten of October 2015 a impeachment request for president Dilma Rousseff was handed to the Chamber Of Representatives' president, who accepted the request on December second. The opening of the impeachment process was accepted by the special impeachment committee on eleven of April 2016.

\end{comment}


\subsection{Methods}

\subsubsection{Generalized Autoregressive Conditional Heteroskedastic (GARCH) Model}

The Generalized Autoregressive Conditional Heteroskedastic (GARCH) model, introduced in \citet{bollerslev1986}, is a generalization of the Autoregressive Conditional Heteroskedastic (ARCH) model introduced in \citet{engle1982}, which models heteroscedasticity. While the most common models for time series assume a constant variance for the process (as ARMA and VAR)\todo{referencias}, the ARCH process assumes an inconstant variance conditional on past variance, together with a constant unconditional variance. The GARCH process assumes inconstant variance conditional on past conditional variance as well as past variance. Both ARCH and GARCH are serial uncorrelated processes with zero mean.\todo{ver se o modelo que eu não apliquei não tem média condicional. alias, posso fazer uma terceira analise np em que a media e variancia do CDS se dao por uma regressao arma garch}

Let $y_t$ denote a real-valued discrete time process and $\psi_t$ denote the information set at time $t$, the ARCH process is given by Equation \ref{eqn:archgarch} and Equation \ref{eqn:arch} while the GARCH process is given by Equation \ref{eqn:archgarch} and Equation \ref{eqn:garch}.

\begin{equation}
\label{eqn:archgarch}
y_t | \psi{t-1} \sim N(0,h_t)
\end{equation}
\todo{trocar para p, esta assim no artigo do engle}
\begin{equation}
\label{eqn:arch}
h_t = \alpha_0 + \displaystyle\sum_{i=1}^{q} \alpha_i y_{t-i}^2
\end{equation}

\begin{equation}
\label{eqn:garch}
h_t = \alpha_0 + \displaystyle\sum_{i=1}^{q} \alpha_i y_{t-i}^2 + \displaystyle\sum_{i=1}^{p} \beta_i h_{t-i}
\end{equation}

where

$$q > 0, \quad   p \geq 0$$

$$\alpha_0 > 0, \quad  \alpha_i \geq 0, \quad  i=1,...,q$$

$$\beta_i \geq 0, \quad  i=1,...,p$$

As we can see above, if $p = 0$ it becomes a ARCH process. If $q = 0$ the process is white noise. We will use $ht$ as a estimate for the exchange coupon's standard deviation.







\subsubsection{Estimation}

We estimate a GARCH model for both measures of the exchange coupon. The model estimated will be a GARCH(1,1), therefore we must visually check if one lag for both the auto regressive and the moving average parts of the model is appropriate. We do so by visually inspecting the auto-correlation and partial auto-correlation functions. The graphs for the auto-correlation function and for the partial auto-correlation function are given in Figure \ref{fig:ocacf} and Figure \ref{fig:ocpacf}, respectively, for the OC1 Exchange Coupon, and in Figure \ref{fig:diacf} and Figure \ref{fig:dipacf} for the DI1 Exchange Coupon.

\subfile{graphstext/ocacf.txt}

\subfile{graphstext/ocpacf.txt}

\subfile{graphstext/diacf.txt}

\subfile{graphstext/dipacf.txt}

The visual inspection show support for the use of GARCH(1,1). After the fitting of the model, the residuals from must behave like white noise, that is, the mean must be constant and equal to zero, and there must be no auto-correlation in the series. First, we inspect this behavior visually with the residuals graph and auto-correlation function. The graph for the residuals are in Figure \ref{fig:ocres} for the OC1 measure, in Figure \ref{fig:dires} for the DI1 measure. The graph for the auto-correlation function is in Figure \ref{fig:ocresacf} for the OC1 Exchange Coupon and in Figure \ref{fig:diresacf} for the DI1 Exchange Coupon.

\subfile{graphstext/ocres.txt}

\subfile{graphstext/ocresacf.txt}

\subfile{graphstext/dires.txt}

\subfile{graphstext/diresacf.txt}

To certify white noise behaviour, we test the null hypothesis that the residuals are independently distributed with the Ljung-Box test (\citet{boxpierce} and \citet{ljungbox}), and the null hypothesis that the residuals sample comes from a normal distributed population with the Shapiro-Wilk test \citet{shapirowilk}. The results for the tests, for both measures of the Exchange Coupon, are shown in Table \ref{tab:reswhite}.

\subfile{tables/reswhite.txt}

The results shown in Table \ref{tab:residualswhite} support the hypothesis that the residuals behave as white noise. The Ljung-Box tests cannot reject the null hypothesis that the residuals are independently distributed, while the Shapiro-Wilk tests accepted the null hypothesis that the residuals sample comes from a normal distributed population.

\subsubsection{Volatility Estimate}

As measure of volatility we use the Conditional Standard Deviation (CSD) series extracted from the GARCH model. The CDS series are shown in Figure \ref{fig:occsd} for the OC1 measure, and in Figure \ref{fig:dicsd} for the DI1 measure. The descriptive statistics for the series are shown in Table \ref{tab:descsd}

\subfile{graphstext/occsd.txt}

\subfile{graphstext/dicsd.txt}

\subfile{tables/descsd.txt}

We consider abnormal volatility every value outside the 95\% confidence interval. We used both a  and a non parametric analysis to filter for abnormal volatility. In the former, we assume a two-parameter distribution for the population, while in the latter we do not make this assumption.

\subsubsection{Parametric}

In the parametric analysis, we assume a two parameter distribution for the Conditional Standard Deviation series when computing the 95\% confidence interval that will be used to filter the series for abnormal volatility. The upper and lower limits are defined in Equations \ref{eqn:upp} and \ref{eqn:lop}: 


$$\bar{X} = \frac{1}{n} \displaystyle\sum_{i=0}^{n} CSD_t$$

$$\sigma^2 = \frac{1}{n-1} \displaystyle\sum_{i=0}^{n} (CSD_t - \bar{X})^2$$

$$\sigma = \sqrt{\sigma^2}$$

\begin{equation}
\label{eqn:upp}
UpperLimit_t = \bar{X} + 1.96 * \sigma
\end{equation}

\begin{equation}
\label{eqn:lop}
LowerLimit_t = \bar{X} - 1.96 * \sigma
\end{equation}


We test the the null hypothesis that the Conditional Standard Deviation samples come from a normal distributed population with the Shapiro-Wilk test, which results are shown in Table \ref{tab:csdshapiro}.

\subfile{tables/csdshapiro.txt}

The test results shown in Table \ref{tab:csdshapiro} support the null hypothesis.

The lower and upper limits are shown in Table \ref{tab:limpar} for both measures of the Exchange Coupon. Figure \ref{fig:oclimpar} and Figure \ref{fig:dilimpari} show the lower and lower limits altogether with the CDS series, for the OC1 and DI1 Exchange Coupons respectively.

\subfile{tables/limpar.txt}

\subfile{graphstext/oclimpar.txt}

\subfile{graphstext/dilimpar.txt}

Table \ref{tab:ocparout} and Table \ref{tab:diparout} show the details of each day with abnormal volatility, that is, the days in which the conditional standard deviation was outside the limits of the 95\% confidence interval, for the OC1 Exchange Coupon and for the DI1 Exchange Coupon respectively.

\subfile{tables/ocparout.txt}

\subfile{tables/diparout.txt}

 \todo{colocar probabilidade de ter acontecido, para cada ponto de volatilidade anormal ||| transformar essas tabelas também em longtables}

As we see in the tables above, the results are basically the same for both measures of the exchange coupon. There were 18 days of abnormal volatility in our 685 days sample, of which 4 are from 2016, 5 are from 2017, 8 are from 2018, and 1 is from 2019. Of these 18 days, in 10 the exchange coupons had positive values.

\subsubsection{Non Parametric}

While in the parametric analysis we assumed a normal distribution for the series, in the non-parametric analysis we do not make this assumption. We calculated the mean and standard deviation for each day with a 63 days window (22 days before and 22 days after), which corresponds to three months of data. \todo{por que da definicao de 63 dias}. For the days at the beginning and end of the sample for which there were not 22 days before, or after, available, the calculation was made with whatever days were available. The definition of the 63 days windows was made based on the frequency of Copon meetings (one for every one and a half month), in which the goal for the Selic rate is defined, affecting directly both the rate of Selic and the inter-bank deposit rate. The window is twice the period between meetings, a quarter of year.

We define the limits of the non-parametric analysis in Equations \ref{eqn:upnp} and \ref{eqn:lonp}.

$$\bar{X}_t = \frac{1}{63} \displaystyle\sum_{i=t-22}^{t+22} CSD_t$$

$$\sigma^2_t = \frac{1}{62} \displaystyle\sum_{i=t-22}^{t+22} (CSD_t - \bar{X}_t)^2$$

$$\sigma_t = \sqrt{\sigma^2t}$$

\begin{equation}
\label{eqn:upnp}
UpperLimit_t = \bar{X}_t + 1.96 * \sigma_t
\end{equation}

\begin{equation}
\label{eqn:lonp}
UpperLimit_t = \bar{X}_t - 1.96 * \sigma_t
\end{equation}

\begin{comment}
\begin{equation}
\label{eqn:upnp}
UpperLimitt = \frac{1}{63} \displaystyle\sum_{i=t-22}^{t+22} Volatilityt + 1.96 * \Big(\frac{1}{62} \displaystyle\sum_{i=t-22}^{t+22} (Volatilityt - \frac{1}{63} \displaystyle\sum_{i=t-22}^{t+22} Volatilityt)^2\Big)^{1/2}
\end{equation}

\begin{equation}
\label{eqn:lonp}
LowerLimitt = \frac{1}{63} \displaystyle\sum_{i=t-22}^{t+22} Volatilityt - 1.96 * \Big(\frac{1}{62} \displaystyle\sum_{i=t-22}^{t+22} (Volatilityt - \frac{1}{63} \displaystyle\sum_{i=t-22}^{t+22} Volatilityt)^2\Big)^{1/2}
\end{equation}
\end{comment}

\todo{checar se é realmente esse o calculo feito pela funcao que usei no programa}

The lower and upper limits are shown in Table \ref{tab:limnon} for both measures of the Exchange Coupon. Figure \ref{fig:oclimnon} and Figure \ref{fig:dilimnon} show the lower and lower limits altogether with the CDS series, for the OC1 and DI1 Exchange Coupons respectively.

\subfile{tables/limnon.txt}

\subfile{graphstext/oclimnon.txt}

\subfile{graphstext/dilimnon.txt}

Table \ref{tab:ocnonout} and Table \ref{tab:dinonout} show the details of each day with abnormal volatility, that is, the days in which the conditional standard deviation was outside the limits of the 95\% confidence interval, for the OC1 Exchange Coupon and for the DI1 Exchange Coupon respectively.

\subfile{tables/ocnonout.txt}

\subfile{tables/dinonout.txt}

As we see in the tables above, the results are basically the same for both measures of the exchange coupon. There were 28 days of abnormal volatility in our 685 days sample, of which 4 are from 2016, 8 are from 2017, 10 are from 2018, and 6 are from 2019. Of these 28 days, in 16 the exchange coupons had positive values.

\section{Results and discussion}

After having filtered the exchange coupon for abnormal volatility, and the news for political events that could have affected the exchange coupon, we match the two samples to find what political events happened to occur in the days of abnormal volatility. With this information we can analyze the impact of political news in the exchange coupon.

The days of abnormal volatility for both measures of the exchange coupon were the same in each type of analysis (parametric and non parametric), therefore we will talk about the results as being the same for both measures, without referencing a specific measure.

\subsection{Parametric}

The political news for each day of abnormal volatility, for the parametric analysis, are shown in Table \ref{tab:parnewsr}. There are a total of 61 news in 14 days of abnormal volatility.

\subfile{tables/parnews.txt}

The parametric analysis shows 4 periods of abnormal volatility. The first period started at April 19, 2017 with the corruption scandal of president Michel Temer, and lasted until April 25th - a whole week. The scandal begun with the disclosure by the Supreme Court of a recording of a conversation between Mr. Temer and the businessman Joesley Batista - owner of JBS, which was, and still is by the time this is written, the largest meat processing company in the world -, who delivered the recording to the authorities as part of a plea bargain. The day in which the recording was revealed was commonly dubbed as 'Joesley Day', as it was a day of of high volatility both in the stock and in the dollar markets. In fact, the Conditional Standard Deviation for both measures of exchange coupon was of 0.023, the mean plus 13,72 times the standard deviation.

The second period was between June 11th and 13th, 2019. \todo{descrever o que aconteceu no dia}.

The third period lasted just a day, the 23th of July, 2018. In this date, Geraldo Alckmin, a presidential candidate, revealed the candidate for his vice presidency; and Dias Toffoli, a Supreme Court minister, substituted Cármem Lúcia as the president of the Court while she was substituting the country's president.

The last period of abnormal volatility begun 6 days before the 2018 presidential elections, and ended two days after the election. At October 4th, when the period of abnormal volatility started, a voter intent survey was released which result was, in the case of a second round between Jair Bolsonaro and Fernanando Haddad, of 43\% of votes for the first and 42\% for the latter. \todo{comentar sobre caracteristicas diferente dos dois canditatos, no sentido de seus programas de governo, talvez principalmente da atitude em relação à crise fiscal}

After knowing the political news that match with the abnormal volatility days, we can ----- which news actually had an effect in the exchange coupon. We considered that only matters related to changes in the presidency affected the exchange coupon. Both the arise of the possibility of a impeachment of president Temer caused by a corruption scandal and the presidential election with two very different \todo{outra palavra que nao 'diferente'} leading candidates brought abnormal volatility. These periods are shown in Table \ref{tab:respar}.

\begin{table}[H]
\caption{Periods of Abnormal Volatility related to Political News, by Parametric Analysis}
\label{tab:respar}
\centering
\begin{tabular}{| c | c |}
\hline
Period & News Topics \\
\hline \hline
19/05/17 - 25/05/17 & President Temer's corruption scandal \\
\hline
04/10/18 - 10/10/18 & Presidential Elections \\
\hline
\end{tabular}
\end{table}

\todo{11-13/06/18 and 23/07/18}

\subsection{Non Parametric}

The political news for each day of abnormal volatility, for the non parametric analysis, are shown in Table \ref{tab:nonnews}. There are a total of 148 news in 25 days of abnormal volatility.

\subfile{tables/nonnews.txt}

The 25 days of abnormal volatility are divided in 14 periods.

17/03/17 / At March 17, 2017, there was a news about the shrinkage of contractors.

19/05/17 - 23/05/17 and 17/08/17 and 25/10/17 / The three days of abnormal volatility between April 19 and 23, 2017, saw the beginning of the corruption scandal of president Temer, as mentioned in the previous section. Abnormal volatility returned at August 17, when the Federal Council of the Brazilian Lawyers Association (OAB) triggered the Supreme Court to make the president of the Chamber of Representatives analyze the impeachment requests he had received (there were 25 at the time). The impeachment matter caused abnormal volatility for the last time at October 25, when the Chamber of Representatives vote for the corruption denunciation against president Temer to not be sent to the Supreme Court, where it would be judged.

01/12/17 / that the president of the Party of the Republic (PR) used a apartment of the Chamber of Deputies while fugitive (he was convicted for corruption)

26/01/18 - 29/01/18 / News about corruption being more decisive than economics in the presidential elections.

11/06/18 - 13/06/18 /

23/07/18 / Bolsonaro e Boulos sao confirmados como candidatos a presidente.

04/10/18 - 10/10/18 / As in the parametric analysis, the presidential elections brought abnormal volatility in the non parametric analysis, for the same period.

04/01/19 / The forth day of Bolsonaro's presidency was the first day of the his presidency to have abnormal volatility. At the time, Bolsonaro talked about trying to pass a new Pension Reform rather than working on the approval of the reform from the previous president, Temer.

07/01/19 - 08/01/19 | 01/02/19 / From January 7 to January 8, both PSDB, PR and Podemos announced support for the reelecion of Rodrigo Maia for the Chamber of Deputies' presidency. The reelection occurred at February 1st, the date with most political news - 34-, and caused abnormal volatility.

25/03/19 - 28/03/19 / Chamber of Representative's president, Rodrigo Maia, said president Bolsonaro would the political articulation for the Pension Reform, and three days later the rapporteur for reform was chosen.

01/04/19 - 02/04/19 / seg / Plateau Palace, house of president Bolsonaro, share a video negating the denomination 'coup' for the military movement of 1964 which changed the government.

The results from the non parametric analysis add to the results of the parametric analysis, as they show that not only matters related to changes in the presidency affected the exchange coupon, but matters related to the presidency of the chamber of deputies and the pension reform also did. The possibility of impeachment of president Michel Temer impacted the exchange coupon not only in May 2017, but in October as well, when the denunciation against the president was rejected by the chamber of deputies. The elections for the presidency of the chamber of deputies, that took place in February 2019, also brought abnormal volatility. The last political event that seems to have had an effect on the exchange coupon was the definition of the rapporteur of the Pension Reform. The periods of abnormal volatility related to political news, by the parametric analysis, are shown in Table \ref{tab:resnon}.

\begin{table}[H]
\caption{Periods of Abnormal Volatility related to Political News, by Non Parametric Analysis}
\label{tab:resnon}
\centering
\begin{tabular}{| c | c |}
\hline
Period & News Topics \\
\hline \hline
19/05/17 - 23/05/17 & President Temer's corruption scandal and impeachment requests \\
\hline
25/10/17 & Rejection of the denunciation against president Temer \\
\hline
04/10/18 - 10/10/18 & Presidential elections \\
\hline
01/02/19 & Election for the presidency of the Chamber of Deputies \\
\hline
28/03/19 & Definition of the rapporteur of the Pension Reform \\
\hline
\end{tabular}
\end{table}

\section{Conclusion}

In order to test the market for semi-strong form efficiency, we analyzed the impact of political news from a Brazilian online newspaper on the country's exchange coupon. We crossed the days with abnormal returns for the exchange coupon with the days with political news. Ir order to do that, we used web scrapping to search for political news, and applied a GARCH filter in the exchange coupon to filter the series for abnormal returns. We did both a parametric and a non parametric analysis.

The results, from both the parametric and non parametric analysis, show support for the semi-strong form efficient market hypothesis. From 855 days with political news, only a small fraction had effect over the exchange coupon (14 days in the case of the parametric analysis, and 25 days in the case of the non parametric analysis).

The parametric and non parametric analysis differed in their results, but both indicate that the exchange coupon was affected by political news about changes in the presidency, whether they were about a certain change, as in the case of the 2018 elections, or about the possibility of a change caused by a impeachment of the president, as in the case of the corruption scandal with president Michel Temer. These results follow \cite{smales2015} and \cite{marquessantos2016}, as the results of both follow that political uncertainty, as with who would win the elections, causes market uncertainty.

While the parametric analysis indicates that the relationship between political news and the exchange coupon is limited to the one mentioned above, the non parametric analysis indicates that this relationship goes further, as political news of less (but in no way small) significance also affect the exchange coupon. By the non parametric analysis, the election for the presidency of the Chamber of Deputies and the definition of the rapporteur of the Pension Reform also affected the exchange coupon.


\bibliographystyle{apacite}
\bibliography{infassetreturns}

\begin{comment}

The best forecast for the asset price in the next period is always the current price. (quem disse isso? acho que propio Fama)
The best estimate for the asset return in the long term is always the normal return for the risk of the asset (quem fisse isso?)

\end{comment}

\end{document}